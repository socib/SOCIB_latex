\documentclass[draft]{beamer}
\mode<presentation>
{
  \usetheme{SOCIB2015b}
}
\usepackage{amsmath,amssymb}
\usepackage{sfmath} % for sans serif math fonts; wget http://dtrx.de/od/tex/sfmath.sty
\usepackage[english]{babel}
\usepackage[orientation=landscape,scale=0.95]{beamerposter}
\usepackage{booktabs,array}
\usepackage{listings}
\usepackage{xspace}
%\usepackage{fp}
\usepackage{doi}
\usepackage{url}
\usepackage{ragged2e}
\usepackage[justification=centering]{caption}
\usepackage{enumitem}
\usepackage{tikz}
\usetikzlibrary{fadings,shapes.arrows,shadows}   
\usepackage[framemethod=tikz]{mdframed}


\renewcommand{\captionfont}{\it \small}
\urlstyle{sf}
\parskip 5cm

\hypersetup{bookmarksopen=true,
bookmarksnumbered=true,  
pdffitwindow=false, 
pdfstartview=FitH,
pdftoolbar=true,
pdfmenubar=true,
pdfwindowui=true,
pdfauthor={SOCIB Data Centre},
pdftitle={Oceanographic data at your fingertips: the SOCIB App for smartphones},
pdfsubject={},
pdfkeywords={SOCIB{,} Dapp{,} lw4nc2{,} NetCDF},
bookmarksopenlevel=1,
colorlinks=true,urlcolor=bluesocib,
citecolor=blue}

\newcommand{\figwidth}{.2\textwidth}

\newcommand{\icon}[1]{\includegraphics[height=1.5cm]{#1}}
\newcommand{\iconb}[1]{\includegraphics[height=2.5cm]{#1}}
\newcommand{\iconc}[1]{\includegraphics[height=1.75cm]{#1}}

\newcommand{\important}[1]{\textcolor{bluesocib}{\it #1}}

\newcommand{\sepblock}{\vspace*{-1cm}\rule{\columnwidth}{.2cm}}
\newcommand{\sepblocksub}{\vspace*{-1.25cm}\rule{\columnwidth}{.1cm}}
\newcommand{\numbox}[1]{\raisebox{-1.25cm}{\fcolorbox{bluesocib}{bluesocib}{\textcolor{white}{\,\raisebox{1cm}{\fontsize{4.cm}{1cm} #1}\,\rule{0cm}{3cm}}}}}

\newcommand{\subsecnum}[1]{\tikz[baseline=(char.base)]{%
            \node[shape=circle,fill=white,draw=bluesocib,inner sep=3pt] (char) {\textcolor{black}{#1}};}}
            
\newcommand{\subsecnumsmall}[1]{\tikz[baseline=(char.base)]{%
            \node[shape=circle,fill=white,draw=bluesocib,inner sep=1pt] (char) {\textcolor{black}{#1}};}}
            

\newcommand{\users}[1]{\textbf{\textcolor{bluesocib}{Users:}} \textit{#1}}

\NewDocumentEnvironment{vertline}{O{0.25em} O{-1cm} O{white}}%
  {% \begin{coderule}[<rule width>][<rule sep>][<rule colour>]  
    \begin{mdframed}%
      [topline=false,rightline=false,bottomline=false,%
       innertopmargin=5pt,innerrightmargin=0pt,innerbottommargin=5pt,%
       skipabove=\parskip,skipbelow=0.3\baselineskip,%
       innerleftmargin=#2,outerlinewidth=#1,linecolor=#3,backgroundcolor=blacksocib]
  }
  {\end{mdframed}}% \end{coderule}
  

\newenvironment{enumline}{
	\begin{vertline}\begin{enumerate}[label=---\protect\subsecnum{\arabic*},font=\sffamily,leftmargin=*,labelindent=1cm,before=\color{white}]
}{
	\end{enumerate}\end{vertline}
}

\newenvironment{enumlineB}{
	\begin{vertline}\begin{enumerate}[label=---\protect$\bullet$,font=\sffamily,leftmargin=*,labelindent=1cm,before=\color{white}]
}{
	\end{enumerate}\end{vertline}
}


\listfiles
\graphicspath{{/home/ctroupin/Presentations/figures4presentations/icons/},{/home/ctroupin/Presentations/figures4presentations/logo/},{/home/ctroupin/Presentations/figures4presentations/app/},
{./facilities_icons/}}

% Display a grid to help align images
%\beamertemplategridbackground[1cm]

\title[SOCIB App for smartphones]{Oceanographic data at your fingertips: the SOCIB App for smartphones}

\author{S.~Lora, K. Sebasti\'{a}n, C.~Troupin, J.P.~Beltran, B.~Frontera, S. G\'{o}mara and J. Tintor\'{e}}
\institute{SOCIB, IMEDEA}

\date{\today}


\begin{document}
\begin{frame}{} 
\justifying
%\vspace{-1cm}
\rule{\columnwidth}{.2cm}


  
\begin{columns}[T]
  
% 1st column %-------------------------------------------------------------------------------------------------------------------------------------------------------------------------
\begin{column}{.32\linewidth}
    
    
\begin{block}{\numbox{1}~What is SOCIB?}
\justifying
SOCIB is a multi-platform  Marine Research Infrastructure located in the Mediterranean Sea and made up of various facilities:
\begin{enumline}
\item the coastal research vessel \icon{vessel}
\item the HF radar \icon{hf_radar}
\item the gliders \icon{glider}
\item the Lagrangian platforms (profiling buoys \icon{profiler_drifter} and surface drifters \icon{surface_drifter}) 
\item the fixed stations (oceanographic buoys \icon{oceanographic_buoy}, weather \icon{weather_station}\\ 
and sea level stations \icon{sea_level} ) 
\item the beach monitoring (beach cameras \icon{beach_monitoring}\, and coastal stations \icon{coastal_station}) 
\item the modelling and forecast (waves and hydrodynamics) \icon{forecast}
\end{enumline}
The goal is to provide access to all the data generated by these facilities through an user-friendly \textit{app} running on smartphones.
\vspace*{.5cm}
\end{block}

\sepblock

\begin{block}{\numbox{2}~Technology}
\justifying
The app relies on the SOCIB Data Discovery Service (\url{http://apps.socib.es/DataDiscovery/index.jsp}), a layer of \textit{RESTful} web services.

\begin{figure}
\centering
\parbox{.55\textwidth}{
These web services allow the client to obtain:

\vspace{.2cm}

\begin{enumlineB}
\item a list of fixed stations or deployments\\ (active and archived),
\item measurements for a given platform or a selected variable,
\item a time series for a given data product.
\end{enumlineB}
}\parbox{.02\textwidth}{\hspace{.5cm}
}\parbox{.42\textwidth}{
\centering
\includegraphics[width=\figwidth]{appposter_home}\,\includegraphics[width=\figwidth]{appposter_menu1}
}
\end{figure}

\end{block}

\sepblock

\begin{block}{\numbox{3}~Facilities}
\justifying
All the data acquired by SOCIB facilities are presented with the platform locations and the most recent measurements.
\end{block}

\begin{block}{\subsecnum{\iconb{weather_station}}~Fixed stations}
\justifying
\begin{figure}
\centering
\parbox{.34\textwidth}{
Current observations from different fixed stations, as well as time series for the last day, week or month are provided.

The user can easily share the data in social media and save his favourite variables.
}\parbox{.01\textwidth}{\,
}\parbox{.65\textwidth}{
\flushright
\includegraphics[width=\figwidth]{appposter_fixedstation2}\,\includegraphics[width=\figwidth]{appposter_fixedstation3}\,\includegraphics[width=\figwidth]{appposter_fixedstation4}
}
\end{figure}
\end{block}



\end{column}

% 2nd column
%-------------------------------------------------------------------------------

\begin{column}{.32\linewidth}
   
\begin{block}{\subsecnum{\iconb{glider}}~Gliders}
\justifying
\begin{figure}
\centering
\parbox{.415\textwidth}{
All the glider missions performed by SOCIB are listed. On the map, the way-points and the current positions can be visualised.

The figure shows an example of two missions performed in November 2014.}\parbox{.01\textwidth}{\,
}\parbox{.575\textwidth}{
\centering
\includegraphics[width=\figwidth]{appposter_glider2}\,\includegraphics[width=\figwidth]{appposter_glider3}
}
\end{figure}
\end{block}

\begin{block}{\subsecnum{\iconb{profiler_drifter}}~Lagrangian plaforms}
\justifying
\begin{figure}
\centering
\parbox{.415\textwidth}{
Similarly to the gliders, the drifters and profilers can be displayed on the map. Only the positions for the last 30~days are shown.

When the user taps on a buoy, a window with additional information opens. 
}\parbox{.01\textwidth}{\,
}\parbox{.575\textwidth}{
\centering
\includegraphics[width=\figwidth]{appposter_drifters}\,\includegraphics[width=\figwidth]{appposter_drifters3}
}
\end{figure}
\end{block}

\begin{block}{\subsecnum{\iconb{beach_monitoring}}~Beach monitoring}
\justifying
\begin{figure}
\centering
\parbox{.415\textwidth}{
Real-time images are provided for Playa de Palma, Calla Millor and Son Bou (see map for locations).

The image shows a snapshot at Calla Millor on March~24, 2015.
}\parbox{.01\textwidth}{\,
}\parbox{.575\textwidth}{
\centering
\includegraphics[width=\figwidth]{appposter_beachmonitoring1}\,\includegraphics[width=\figwidth]{appposter_beachmonitoring2}

}
\end{figure}
\end{block}

\vspace{1.5cm}

\sepblock

\begin{block}{\numbox{4}~Where to get the app?}
\justifying
\vspace{.5cm}
\parbox{.475\columnwidth}{
\includegraphics[width=6cm]{logo_android.png}\hspace{1cm}\subsecnumsmall{\iconc{qr_code_android}}\\
\url{https://play.google.com/store/apps/details?id=com.socib}
}\parbox{1cm}{\hspace*{1cm}
}\parbox{.475\columnwidth}{
\includegraphics[width=6cm]{logo_ios.png}\hspace{1cm}\subsecnumsmall{\iconc{qr_code_iOS}}\\
\url{http://itunes.apple.com/es/app/socib/id482542716?mt=8}
}
\vspace*{1cm}

Other applications developed by SOCIB Data Centre are available at\\
\url{http://apps.socib.es/}.

\end{block}

\end{column}


% 3rd column
%-------------------------------------------------------------------------------

\begin{column}{.32\linewidth}

\begin{block}{\subsecnum{\iconb{vessel}}~Research Vessel}
\justifying
\begin{figure}
\centering
\parbox{.415\textwidth}{
All the missions performed by SOCIB Research Vessel can be visualised. If a mission is on-going, the positions are shown in near-real time.

The \textit{Alborex} mission (May 2014) south of Cartagena is shown as an example.
}\parbox{.01\textwidth}{\,
}\parbox{.575\textwidth}{
\centering
\includegraphics[width=\figwidth]{appposter_ship}
}
\end{figure}
\end{block}

\begin{block}{\subsecnum{\iconb{forecast}}~Forecast}
\justifying
\begin{figure}
\centering
\parbox{.415\textwidth}{
Wave and hydrodynamic models provide 3-day forecasts of various variables such as: wave height and period, sea water temperature and salinity, currents, \ldots}\parbox{.01\textwidth}{\,
}\parbox{.575\textwidth}{
\centering
\includegraphics[width=\figwidth]{appposter_sapo}\,\includegraphics[width=\figwidth]{appposter_model}
}
\end{figure}
\end{block}

\begin{block}{\subsecnum{\iconb{hf_radar}}~HF Radar}
\justifying
\begin{figure}
\centering
\parbox{.415\textwidth}{
The HF radar system measures the surface currents in the eastern part of the Ibiza Channel. 

The last day of data is provided, with one image per hour.
}\parbox{.01\textwidth}{\,
}\parbox{.575\textwidth}{
\centering
\includegraphics[width=\figwidth]{appposter_radar}
}
\end{figure}
\end{block}

\vspace{1.5cm}

\sepblock
\begin{block}{\numbox{5}~Future work}
\justifying

The objective to provide an easy access to all the data managed by SOCIB through a mobile app has been fulfilled. 

\vspace*{.5cm}

Other apps for specific user profiles (sailors, surfers, tourists) may be developed.

\vspace*{1cm}

Feedback and suggestions can be sent to \hfill {\LARGE \Letter~ \href{mailto:info@socib.es}{info@socib.es}} \hspace{2cm}

\hfill or via \includegraphics[height=1.5cm]{logo_twitter}\, {\LARGE \textcolor{bluesocib}{\symbol{64}SOCIB\_data}} \hspace{2cm}

\end{block}

\end{column}

\end{columns}

\vfill

\end{frame}

\end{document}





